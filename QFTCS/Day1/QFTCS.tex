\documentclass{jsarticle}

\usepackage{bm}
\usepackage{mathtools}
\usepackage{braket}
\usepackage{listings}
\usepackage{jvlisting}
\usepackage{ascmac}
\usepackage{mathrsfs}
\usepackage{amsmath}
\usepackage{amsthm}
\usepackage{color}
\usepackage[dvipdfmx]{graphicx}
\usepackage{amssymb}
\usepackage{amsfonts}
\usepackage{tikz}
\usepackage[dvipdfmx]{hyperref}
\usepackage{pxjahyper}

\newtheorem{thm}{定理}
\newtheorem{dfn}{定義}
\newtheorem{axi}{公理}
\newtheorem{lem}{補題}
\newtheorem{cor}{系}
\newtheorem{prp}{命題}
\newtheorem{exm}{例}
\newtheorem{rem}{注意}



\title{Quantum Field Theory in Curved Spacetime}
\author{三角 矢雲}
\date{\today}

\begin{document}



\maketitle

\section{数学的準備}

この章では多様体と微分幾何学の復習をする.

\subsection{反変ベクトルと共変ベクトル}

まずは,微分についての定義する.

\begin{itembox}[l]{導関数}

ある滑らかな写像$\varphi:\mathbb{R}^n \to \mathbb{R}^m$を考える.このとき線形写像$D\varphi\rvert_x:\mathbb{R}^n \to \mathbb{R}^m$を$v \in \mathbb{R}^n$に対して

\begin{align}
D\varphi\rvert_x (v) :=\lim_{t \to 0} \frac{\varphi(x+tv) - \varphi(x)}{t}
\end{align}

と定義し,これを導関数と呼ぶ.さらにこれは結合則

\begin{align}
D(\varphi \circ \psi)\rvert_x = D\varphi\rvert_{\psi(x)}D\psi\rvert_x
\end{align}

を満たす.

\end{itembox}\\

次に多様体における座標系について定義する.

\begin{itembox}[l]{座標近傍(チャート)と遷移関数}

$M$を$n$次元多様体とし,$V$を$\mathbb{R}^n$の開部分集合とする.\\
$M$の開部分集合$U$と同相写像$\kappa:U \to V$の組$(U,\kappa)$を$M$の座標近傍または,チャートと呼ぶ.\\
また,2つのチャート$(U,\kappa)$と$(U^\prime,\kappa^\prime)$について$U \cap U^\prime \neq\emptyset$であるとき,遷移関数$\varphi:\kappa(U \cap U^\prime) \to \kappa^\prime(U \cap U^\prime)$を次のように定義する.

\begin{align}
\varphi = \kappa^\prime \circ \kappa^{-1}
\end{align}

このとき,写像$\varphi$は滑らかである.

\end{itembox}

$M$上の幾何学的対象はこのチャートを用いて表現することが可能である.チャートが重なっている部分では様々なチャート表現があり,それらは遷移関数を用いて変換される.以下に具体例を挙げる.\\

\begin{itemize}

\item 関数$f:M \to \mathbb{R}$のチャート表現$f_\kappa:\kappa(U) \to \mathbb{R}$は$f_\kappa = f \circ \kappa^{-1}$と与えられる.この変換則はスカラーであり,

\begin{align}
f_{\kappa^\prime}(\varphi(x)) =f_\kappa (x)
\end{align}

と変換される.\\

\item $M$上のベクトル場$X$のチャート表現$X_\kappa:\kappa(U) \to \mathbb{R}^n$は反変的な変換則になり,

\begin{align}
X_{\kappa^\prime}(\varphi(x)) = D\varphi\rvert_x X_\kappa(x)
\end{align}

と変換される.\\

\item $M$上の共変ベクトル場$\xi$のチャート表現$\xi_\kappa:\kappa(U)\to (\mathbb{R}^n)^*$は共変的な変換則となり,

\begin{align}
\xi_{\kappa^\prime}(\varphi(x))D\varphi\rvert_x = \xi_\kappa(x)
\end{align}

と変換される.\\

\item 重み$k$の密度$\rho$はチャート表現$\rho_\kappa:\kappa(U) \to \mathbb{R}$を持ち,

\begin{align}
\rho_{\kappa^\prime}(\varphi(x))|\det D\varphi\rvert_x|^k= \rho_\kappa(x)
\end{align}

という変換則を持つ.\\

\item $c:\mathbb{R} \to M$を滑らかな曲線とする.このとき,$C(t)$での接ベクトル$\dot{c}(t)$は以下のように表される.

\begin{align}
\dot{c}(t)_\kappa=D(\kappa \circ c)\rvert_t=\frac{d}{dt}\kappa(c(t))
\end{align}

また,連鎖律より,チャート間では,接ベクトルは反変的に変換される.

\begin{align}
\dot{c}(t)_{\kappa^\prime} =D(\kappa^\prime \circ c)\rvert_t=D(\varphi \circ \kappa \circ c)=D\varphi\rvert_{\kappa(c(t))}D(\kappa\circ c)= D\varphi\rvert_{\kappa(c(t))}\dot{c}(t)_\kappa
\end{align}

同様に,$f:M\to\mathbb{R}$がなめらかな関数であるとき,共変ベクトル場$\nabla f$は以下のように表される

\begin{align}
(\nabla f)_\kappa =Df_\kappa
\end{align}

これに対しても連鎖率より,

\begin{align}
(\nabla f)_\kappa =D(f_{\kappa^\prime}\circ \varphi)=Df_{\kappa^\prime} D\varphi =(\nabla f)_{\kappa^\prime}D\varphi
\end{align}

共変的に変換されることがわかる.\\

\item 重み1の密度は単に「密度」と呼ばれており,$U \cap U^\prime$を台に持つ密度に対して

\begin{align}
\int_{\kappa(U)}\rho_\kappa(x)d^nx=\int_{\kappa^\prime(U)}\rho_\kappa^\prime(y)d^ny
\end{align}

が成り立つことがわかる.ここから,$\rho$が$U$を台に持つならば,重み付けによって積分を

\begin{align}
\int_M \rho:=\int_{\kappa(U)}\rho_\kappa (x)d^nx
\end{align}

と定義できる.このとき,密度は積分測度を含んでいる.\\

\item $X$をベクトル場,$\xi$を共変ベクトル場とする.この時,スカラー関数$\xi(X)$が存在し,

\begin{align}
\xi(X)_\kappa=\xi_\kappa \cdot X_\kappa
\end{align}

と表される.ただし,$\cdot$は行ベクトルと列ベクトルの積を表す.\\

\end{itemize}

今までチャート表現について述べてきたが,今まで述べた例にはさらに整頓された定式化が存在する.例えば,点$p$におけるベクトルは線形写像$v:C^\infty(M;\mathbb{R})\to\mathbb{R}$によって表現されてライプニッツ則を満たす.

\begin{align}
v(fg)=v(f)g(p)+f(p)v(g)\quad f,g\in C^\infty(M;\mathbb{R})
\end{align}

さらに,以下のような$p$を通る曲線の同値類として表現することもできる.

\begin{align}
c\sim c^\prime \Leftrightarrow D(f\circ c)(0)=D(f \circ c^\prime)(0)
\end{align}

ただし,$c(0)=c^\prime(0)=p$である.

\end{document}